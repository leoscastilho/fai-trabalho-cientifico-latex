\documentclass[a4paper,12pt]{article}  % Tipo de documento como artigo, com tamanho de fonte 12pt e formato A4
%Referência de Modelo para a FAI: https://www.fai-mg.br/biblio/images/publicacoes/modelo/ModeloArtigoCientifico.pdf %

\usepackage[brazil]{babel}  % Usa o idioma português (Brasil) para definir as convenções de idioma, como a separação de sílabas e tradução de termos.
\usepackage[utf8]{inputenc}  % Define a codificação de entrada como UTF-8 para permitir caracteres acentuados corretamente.
\usepackage[T1]{fontenc}  % Define a codificação de saída de fontes como T1, permitindo o uso correto de caracteres especiais.
\usepackage{times}     % Fonte Times New Roman
\usepackage{parskip}   % Remove a identação dos parágrafos
\usepackage{geometry}  % Permite personalizar as margens e o layout da página.
\usepackage{graphicx}  % Permite inserir gráficos, imagens e ajustar o tamanho deles.
\usepackage{titlesec}  % Fornece comandos para personalizar o formato dos títulos das seções.
\usepackage{fancyhdr}  % Permite personalizar os cabeçalhos e rodapés do documento.
\usepackage{setspace}  % Controla o espaçamento entre linhas no documento (aqui é usado para espaçamento de 1,5 linha).
\usepackage{csquotes}  % Fornece suporte para citações, garantindo que o estilo de aspas seja correto em diferentes idiomas.
\usepackage{ragged2e}  % Fornece comandos para justificar ou ajustar o alinhamento de texto.
\usepackage{lipsum}    % Para gerar texto Lorem Ipsum
\usepackage{enumitem} % Customizar alíneas

% Define a numeração da página
\fancypagestyle{numbered}{
    \fancyhf{}
    \fancyhead[R]{{\fontsize{10pt}{12pt}\selectfont\thepage}}
    \renewcommand{\headrulewidth}{0pt}
}

% Quanto à separação entre títulos, subtítulos e texto, observa-se o seguinte:
\let\oldsection\section
\renewcommand{\section}[1]{%
    \clearpage % os títulos de capítulos estão sempre no início de uma nova página;
    \thispagestyle{empty} % No page number on this page
    \oldsection{#1}
}
% seções primárias: os títulos dessas seções devem ser em caixa alta, fonte 12 e em negrito;
\titleformat{\section}{\normalfont\large\bfseries}{\thesection}{0.3em}{\MakeUppercase} 

\let\oldsubsection\subsection
\renewcommand{\subsection}[1]{%
    \par\vspace{\parskip}% Add paragraph break before
    \oldsubsection{#1}
    \par\vspace{\parskip} % separação entre o texto e os subtítulos: pula-se uma linha em branco.
}
% seções secundárias: os títulos dessas seções devem ser iguais aos das primárias, porém sem negrito;
\titleformat{\subsection}{\normalfont}{\thesubsection}{0.3em}{\MakeUppercase} 

% Define spacing for subsubsection
\let\oldsubsubsection\subsubsection
\renewcommand{\subsubsection}[1]{%
    \par\vspace{\parskip}% Add paragraph break before
    \oldsubsubsection{#1}
    \par\vspace{\parskip}% Add paragraph break after
}
% seções terciárias: os títulos dessas seções devem ser em negrito, somente com a primeira letra maiúscula;
\titleformat{\subsubsection}{\normalfont\bfseries}{\thesubsubsection}{0.3em}{}

\geometry{
    left=3cm,
    right=2cm,
    top=3cm,
    bottom=2cm,
    headsep=35pt,
    footskip=0pt
    % , showframe
} % Define as margens da página (esquerda = 3cm, direita = 2cm, topo = 3cm, fundo = 2cm)

\setlength{\parindent}{0pt}  % Paragráfos sem identação, conforme modelo da FAI
% \onehalfspacing  % Espaçamento de 1.5 entre linhas
\setstretch{1.5}  % This also sets 1.5 line spacing
\setlength{\parskip}{12pt}  % Espaçament ode 12pt entre paragrafos

% Configuração das alíneas
\newlist{alphaitemize}{enumerate}{1}
\setlist[alphaitemize,1]{
    label=\alph*), % c) as alíneas devem ser indicadas alfabeticamente, em letra minúscula, seguida de parêntese fechado. Utilizam-se letras dobradas, quando esgotadas as letras do alfabeto;
    leftmargin=1cm,        % Total indentation from current margin
    itemindent=0pt,        % No additional item indentation
    labelsep=0pt,          % No space between label and text
    labelwidth=13pt,       % d) as letras indicativas das alíneas devem apresentar recuo de 0,5 cm em relação à margem esquerda;
    itemsep=0pt,
    parsep=0pt,
    topsep=0pt,
    partopsep=0pt,
    align=left,
    before={\setlength{\parskip}{0pt}},  % h) Entre as alíneas, inclusive a primeira delas, os parágrafos são formatados com zero ponto ante e zero depois
    after={\setlength{\parskip}{12pt}}   % i) O próximo parágrafo, após a última alínea, volta a ter formatação normal com 12 pontos antes e 12 depois.
}

\usepackage[sorting=nyt]{biblatex}  % Carrega o pacote BibLaTeX para bibliografia e define o estilo de ordenação como "nyt" (nome, ano, título).
\renewcommand*{\bibfont}{\raggedright}  % Alinha a bibliografia à esquerda (de forma justificada).

% Redefine o título da seção de referências para maiúsculas e centralizado
\defbibheading{bibliography}{%
  \begin{center}
    \textbf{\MakeUppercase{Referências}}
  \end{center}
}

% Customização do Sumário ----------------------------------------------------------
\usepackage{tocloft}
\usepackage{xstring}
\usepackage{textcase}

\renewcommand{\contentsname}{SUMÁRIO}
\renewcommand{\cftsecfont}{\bfseries}
\renewcommand{\cftsecpagefont}{\bfseries}
\renewcommand{\cftsecleader}{\cftdotfill{\cftdotsep}}
\setlength{\cftbeforetoctitleskip}{0pt}
\setlength{\cftaftertoctitleskip}{0.4cm}
\renewcommand{\cfttoctitlefont}{\hfil\bfseries}


\renewcommand{\cfttoctitlefont}{\hfil\rmfamily\bfseries\MakeUppercase} % Título maiúsculo
\renewcommand{\cftdotsep}{0.1} % Adiciona mais pontos. Menor numero = Mais pontos
\setlength{\cftbeforesecskip}{0em} % Espaço entre as linhas de seção
\setlength{\cftbeforesubsecskip}{0em} % Espaço entre as linhas de subseção
\setlength{\cftbeforesubsubsecskip}{0em} % Espaço entre as linhas de subsubseção
\setlength{\cftsecindent}{0em} % Remove identação das seções
\setlength{\cftsubsecindent}{0em}   % Remove identação das subseções
\setlength{\cftsubsubsecindent}{0em} % Remove identação das subsubseções

\setlength{\cftsecnumwidth}{1.2em} % Espaço entre número da seção e texto
\setlength{\cftsubsecnumwidth}{1.5em} % Espaço entre número da subseção e texto
\setlength{\cftsubsubsecnumwidth}{2.3em} % Espaço entre número da subsubseção e texto

% Seção, subseção e subsubseção em maiúsculo no Sumário
\makeatletter
\let\oldcontentsline\contentsline
\renewcommand{\contentsline}[4]{%
  \expandafter\ifx\csname l@#1\endcsname\l@section
    \oldcontentsline{#1}{\MakeUppercase{#2}}{#3}{#4}%
  \else
    \expandafter\ifx\csname l@#1\endcsname\l@subsection
      \oldcontentsline{#1}{\MakeUppercase{#2}}{#3}{#4}%
    \else
      \expandafter\ifx\csname l@#1\endcsname\l@subsubsection
        \oldcontentsline{#1}{\MakeUppercase{#2}}{#3}{#4}%
      \else
        \oldcontentsline{#1}{#2}{#3}{#4}%
      \fi
    \fi
  \fi
}
\makeatother
% ------------------------------------------------------------------------------------

% Funções customizadas para criar e imprimir autores ---------------------------------
\usepackage{pgffor}

\newcounter{authorcount} % Counter para autores

% Comando para adicionar novos autores
\newcommand{\AddAuthor}[2]{% #1 = nome, #2 = email
    \stepcounter{authorcount}
    \expandafter\def\csname author\theauthorcount\endcsname{#1}
    \expandafter\def\csname email\theauthorcount\endcsname{#2}
}

% Comando para imprimir os autores
\newcommand{\PrintAuthorsLombada}{%
    \foreach \i in {1,...,\theauthorcount} {%
        \textbf{\fontsize{12}{14}\selectfont\MakeUppercase{\csname author\i\endcsname}} \\
    }%
}
% ------------------------------------------------------------------------------------

% Funções para Citação Direta ---------------------------------
\newenvironment{citacaodireta}
{\par\addvspace{\baselineskip}% Add exactly one line space before
 \begingroup
 \list{}{\leftmargin=4cm
         \rightmargin=0pt
         \parsep=0pt
         \itemsep=0pt
         \topsep=0pt}
 \item\relax
 \singlespacing
 \fontsize{10}{12}\selectfont
 \justifying}
{\endlist
 \endgroup
 \par\addvspace{\baselineskip}}% Add exactly one line space after

% Usar:
% \begin{citacaodireta}
% ...
% \end{citacaodireta}
% ------------------------------------------------------------------------------------

% Customização da Lista de Figuras ---------------------------------
\usepackage[labelfont=bf]{caption}
\usepackage[figure]{totalcount} % Add this package
\makeatletter
\renewcommand{\@cftmakeloftitle}{} % Completely remove title generation
\makeatother
% Configure the figure entries
\renewcommand{\cftfigpresnum}{FIGURA }
\setlength{\cftfignumwidth}{2.2cm} % Reduce this width to avoid line breaking
\setlength{\cftfigindent}{0pt} % No indentation
% Remove extra spacing between number and title
\renewcommand{\cftfigpresnum}{FIGURA }
\renewcommand{\cftfigaftersnum}{ - }
% ------------------------------------------------------------------------------------

\addbibresource{bib/references.bib}  % Adiciona o arquivo de bibliografia (bib/references.bib) como fonte para as referências bibliográficas.
 % Definições de formato e estilo do documento seguindo diretrizes da FAI
% ============================================================================
% INFORMAÇÕES DO DOCUMENTO - EDITE APENAS O CONTEÚDO ENTRE {}
%
% thetitle: Título do trabalho
% theinstitution: Instituição onde o trabalho é publicado
% theplace: Cidade - Estado onde o trabalho é publicado
% theyear: Ano de publicação do trabalho
% ============================================================================
\newcommand{\thetitle}{DIRETRIZES PARA ELABORAÇÃO DE TRABALHOS CIENTÍFICOS: PADRÃO ABNT E ADAPTAÇÃO ÀS NORMAS INSTITUCIONAIS DA FAI}
\newcommand{\theinstitution}{FAI - CENTRO DE ENSINO SUPERIOR EM GESTÃO, TECNOLOGIA E EDUCAÇÃO}
\newcommand{\theplace}{SANTA RITA DO SAPUCAÍ - MG}
\newcommand{\theyear}{2023}


% ============================================================================
% AUTORES - ADICIONE OU REMOVA CONFORME NECESSÁRIO
% ============================================================================
% Para adicionar um autor, copie a linha abaixo e mude o conteúdo
% Para remover um autor, delete ou comente (%) a linha correspondente
\AddAuthor{Autor 1}{email1@gmail.com}
\AddAuthor{Autor 2}{email2@gmail.com} % Definições de Título, instituição, autores...
% Definições por grupos de elementos textuais
\newenvironment{ElementosExternos}{
    \pagenumbering{roman}
    \pagestyle{numbered}
}{}

\newenvironment{ElementosPreTextuais}{
    \pagestyle{numbered}
}{}

\newenvironment{ElementosTextuais}{
    \clearpage
    \setcounter{savepage}{\value{page}} % Save current page number
    \pagenumbering{arabic}
    \setcounter{page}{\value{savepage}} % Restore the page number
    \pagestyle{numbered}
}{
    \newpage
}

\newenvironment{ElementosPosTextuais}{
    \clearpage
    \pagestyle{numbered}
}{}

% Definições por tipos de conteúdo
\newenvironment{Capa}{
    \setcounter{page}{1}
    \thispagestyle{empty} % Remove numeração da primeira página
    \begin{center}
}{
    \textbf{\theinstitution}
    
    \vfill
    
    \textbf{\fontsize{12}{14}\selectfont\MakeUppercase{\thetitle}}
    
    \vfill
    \textbf{\theplace}\\
    \textbf{\theyear}
    \end{center}
    \newpage
}

\newenvironment{Lombada}{
    \begin{center}
}{
    \textbf{\theinstitution}
    
    \vfill
    
    \begin{center}
        \PrintAuthorsLombada
    \end{center}
    
    \vfill
    
    \textbf{\fontsize{12}{14}\selectfont\MakeUppercase{\thetitle}}
    
    \vfill
    
    \textbf{\theplace}\\
    \textbf{\theyear}
    \end{center}
    \newpage
}

\newenvironment{FolhaDeRosto}{
    
}{}

\newenvironment{Errata}{
    
}{}

\newenvironment{FolhaDeAprovacao}{
    
}{}

\newenvironment{Dedicatoria}{
    
}{}

\newenvironment{Agradecimentos}{
    \begin{center}
    \textbf{AGRADECIMENTOS}
    \end{center}
    \noindent
}{
    \newpage
}

\newenvironment{Epigrafe}{
    \vspace*{\fill}
    \begin{flushright}
    \begin{minipage}{0.5\textwidth}
    \raggedleft
    \singlespacing
    \justifying
}{
    \end{minipage}
    \end{flushright}
    \vspace{1cm}
    
    \newpage
}

\newenvironment{Resumo}{
    \begin{center}
    \textbf{RESUMO}
    \end{center}
    
    \justifying
    \begin{singlespace}
    \theabstract
    \end{singlespace}
    
    \textbf{Palavras-chave:} \thekeywords
    
    \vspace{1cm}
    \begin{center}
    \textbf{ABSTRACT}
    \end{center}
    
    \justifying
    \begin{singlespace}
    \theabstractenglish
    \end{singlespace}
    
    \textbf{Keywords:} \thekeywordsenglish
}{
    \newpage
}


% Comandos para listas condicionais de ilustrações
% Usando sistema de duas passadas com arquivo auxiliar
\newcommand{\ListaDeFiguras}{%
    \ifhasfiguras{
        \clearpage
        \begin{center}
        \textbf{LISTA DE FIGURAS}
        \vspace{0.3cm}
        \end{center}
        \listoffigura
        \clearpage
    }{}
}

\newcommand{\ListaDeFotografias}{%
    \ifhasfotografias{
        \clearpage
        \begin{center}
        \textbf{LISTA DE FOTOGRAFIAS}
        \vspace{0.3cm}
        \end{center}
        \listoffotografia
        \clearpage
    }{}
}

\newcommand{\ListaDeGraficos}{%
    \ifhasgraficos{
        \clearpage
        \begin{center}
        \textbf{LISTA DE GRÁFICOS}
        \vspace{0.3cm}
        \end{center}
        \listofgrafico
        \clearpage
    }{}
}

\newcommand{\ListaDeQuadros}{%
    \ifhasquadros{
        \clearpage
        \begin{center}
        \textbf{LISTA DE QUADROS}
        \vspace{0.3cm}
        \end{center}
        \listofquadro
        \clearpage
    }{}
}

\newenvironment{ListaDeTabelas}{
    
}{}

% Create a counter for siglas
\newcounter{siglacount}

% Create a temporary storage for siglas
\newtoks\siglacontent

% Modified sigla command that increments counter and stores content
\newcommand{\sigla}[2]{%
  \stepcounter{siglacount}%
  \siglacontent\expandafter{\the\siglacontent #1 \> - \quad #2 \\}%
}

% Modified environment that checks the counter
\newenvironment{ListaAbrevESiglas}{%
  % Reset counter and content at the beginning
  \setcounter{siglacount}{0}%
  \siglacontent{}%
}{%
  % Only display if more than 9 siglas
  \ifnum\value{siglacount}>9
    \clearpage
    \begin{center}
    \textbf{LISTA DE ABREVIATURAS E SIGLAS}
    \end{center}
    \vspace{0.5cm}
    \begin{tabbing}
    \hspace{2cm} \= \kill % Set tab stop at 2.5cm
    \the\siglacontent
    \end{tabbing}
    \clearpage
  \fi
}

\newenvironment{ListaDeSimbolos}{
    
}{}

\newenvironment{Sumario}{
    \clearpage
    \tableofcontents
    \thispagestyle{numbered}
    \clearpage
}{}

\newenvironment{Introducao}{
    
}{}

\newenvironment{Desenvolvimento}{
    
}{}

\newenvironment{Conclusao}{
    
}{}

\newenvironment{Referencias}{
    \thispagestyle{empty} % Remove numeração da primeira página
}{
    \printbibliography
}

\newenvironment{Glossario}{
    
}{}

% ============================================================================
% SISTEMA DE APÊNDICES AUTOMÁTICO
% ============================================================================

\newcommand{\myAppendixHeading}[1]{%
    \vspace{2em}%
    \begin{center}%
    \Large\bfseries #1%
    \end{center}%
    \vspace{1em}%
}

% Contador para apêndices
\newcounter{appendixcounter}
\setcounter{appendixcounter}{0}

% Comando para processar uma entrada de apêndice
\newcommand{\appendixentry}[2]{%
    \IfFileExists{apendices/#1.tex}{%
        \stepcounter{appendixcounter}%
        \clearpage%
        \thispagestyle{numbered}%
        {\noindent\normalsize APÊNDICE \Alph{appendixcounter} -- \MakeUppercase{#2}}%
        \vspace{1em}%
        \addtocontents{toc}{%
            \protect\vspace{0em}%
            \protect\noindent\normalsize APÊNDICE \Alph{appendixcounter} -- \MakeUppercase{#2}%
            \protect\cftdotfill{\cftdotsep}\normalsize\thepage\protect\par%
        }%
        \label{apendice:\Alph{appendixcounter}}%
        \input{apendices/#1}%
    }{%
        \PackageWarning{appendix}{Arquivo apendices/#1.tex não encontrado}%
    }%
}

% Comando para incluir todos os apêndices automaticamente
% Lê a configuração do arquivo appendix_list.tex
\newcommand{\includeallappendices}{%
    \IfFileExists{apendices/appendix_list.tex}{%
        \input{apendices/appendix_list.tex}%
    }{%
        \PackageError{appendix}{Arquivo apendices/appendix_list.tex não encontrado}{%
            Crie o arquivo apendices/appendix_list.tex com a lista de apêndices%
        }%
    }%
}

\newenvironment{Apendice}{
    \includeallappendices
}{}

\newenvironment{Anexo}{
    
}{}

\newenvironment{Indice}{
    
}{} % Formatação das páginas

% ============================================================================
% RESUMO - EDITE O TEXTO ABAIXO ENTRE {}
% ============================================================================
\newcommand{\theabstract}{
Esta é a oitava edição do manual “Diretrizes para elaboração de trabalhos científicos: padrão ABNT e adaptação às normas institucionais da FAI”, onde se encontram as normas e os procedimentos para a produção acadêmica da Instituição. Este manual contém um conjunto de recomendações necessárias à elaboração de textos acadêmicos de acordo com as normas da

Associação Brasileira de Normas Técnicas (ABNT). Para aqueles itens que as normas não estabelecem critérios, estabeleceram-se procedimentos próprios e pertinentes à comunidade acadêmica da FAI. Em suas edições anteriores, contou-se com a colaboração dos professores e das bibliotecárias da FAI. As particularidades de cada área do conhecimento presentes nos vários cursos, cada qual com seu estilo, foram trabalhadas para se obter a concisão de ideias e estabelecer o padrão adotado pela FAI como um todo.
}

% ============================================================================
% PALAVRAS-CHAVE - EDITE O TEXTO ABAIXO ENTRE {} (separe com ponto e espaço)
% ============================================================================
\newcommand{\thekeywords}{
Normalização. Procedimentos. Diretrizes. ABNT.
}

% ============================================================================
% RESUMO EM INGLÊS - EDITE O TEXTO ABAIXO ENTRE {}
% ============================================================================
\newcommand{\theabstractenglish}{
This is the eighth edition of the manual “Guidelines for the preparation of scientific papers: ABNT standard and adaptation to FAI institutional norms”, which contains the norms and procedures for the Institution's academic production. This manual contains a set of recommendations necessary for the preparation of academic texts in accordance with the norms of the Associação Brasileira de Normas Técnicas (ABNT). For those items that the norms do not establish criteria, proper procedures were established and pertinent to the academic community of the FAI. In previous editions, FAI teachers and librarians collaborated. The particularities of each area of knowledge present in the various courses, each with its own style, were worked on to obtain concise ideas and establish the standard adopted by the FAI as a whole.
}
% ============================================================================
% PALAVRAS-CHAVE EM INGLÊS - EDITE O TEXTO ABAIXO ENTRE {} (separe com ponto e espaço)
% ============================================================================
\newcommand{\thekeywordsenglish}{
Standardization. Procedures. Guidelines. ABNT.
}

\begin{document}
\newcounter{savepage}

% ============================================================================
% (NÃO EDITAR) CONFIGURAÇÃO DOS ELEMENTOS EXTERNOS SEGUNDO DIRETRIZES DA FAI
% ============================================================================
\begin{ElementosExternos}
\begin{Capa} \end{Capa}
\begin{Lombada} \end{Lombada}
\end{ElementosExternos}
% =============================================================================

% ============================================================================
% (NÃO EDITAR) CONFIGURAÇÃO DOS ELEMENTOS PRÉ-TEXTUAIS SEGUNDO DIRETRIZES DA FAI
% ============================================================================
\begin{ElementosPreTextuais}

\begin{FolhaDeRosto} % --------------------------------------------------------
\end{FolhaDeRosto}

\begin{Errata} % --------------------------------------------------------------
\end{Errata}

\begin{FolhaDeAprovacao} % ----------------------------------------------------
\end{FolhaDeAprovacao}

\begin{Dedicatoria} % ---------------------------------------------------------
\end{Dedicatoria}

% ============================================================================
% AGRADECIMENTOS - EDITE O TEXTO ABAIXO (Entre aspas, com referência ao autor)
% ============================================================================
\begin{Agradecimentos} % -----------------------------------------------------
Aos alunos, professores, colaboradores e revisores pelo apoio e pelas discussões que muito enriqueceram e contribuíram para a realização deste trabalho.

Agradecimento especial ao acadêmico do I e II períodos de 2017, Michel Liberato de Sousa, do Curso de Administração, que revisou todo o documento no final daquele ano, por iniciativa própria, baseado no conteúdo nele contido e no aprendizado obtido na disciplina de Metodologia da Pesquisa Científica, cursada no I período.
\end{Agradecimentos}

% ============================================================================
% EPÍGRAFE - EDITE O TEXTO ABAIXO (Entre aspas, com referência ao autor)
% ============================================================================
\begin{Epigrafe} % -----------------------------------------------------------
“O cientista não só tem que fazer ciência, mas também escrevê-la” (DAY, 1990).
\end{Epigrafe}

\begin{Resumo} % -------------------------------------------------------------
\end{Resumo}

\begin{ListaDeIlustracoes} % -------------------------------------------------
\end{ListaDeIlustracoes}

% ============================================================================
% (NÃO EDITAR) LISTA DE FIGURAS. SOMENTE DISPONÍVEL CASO HAJA FIGURAS
% ============================================================================
\ListaDeFiguras

\begin{ListaDeTabelas} % -----------------------------------------------------
\end{ListaDeTabelas}

% ============================================================================
% ABREVIAÇÕES - ADICIONE ABREVIAÇÕES ABAIXO NO FORMAT \sigla{ABREVIAÇÃO}{DESCRIÇÃO}
%    Só irá mostrar a página caso haja mais que 9 siglas
% ============================================================================
\begin{ListaAbrevESiglas} % ---------------------------------------------------
\sigla{ABNT}{Associação Brasileira de Normas Técnicas}
\sigla{FAI}{Centro de Ensino Superior em Gestão, Tecnologia e Educação}
\sigla{IBGE}{Instituto Brasileiro de Geografia e Estatística}
\sigla{ISBN}{Número Internacional Normalizado para Livros}
\sigla{ISSN}{Número Internacional Normalizado para Publicações Seriadas}
\sigla{JSP}{Java Server Pages}
\sigla{NBR}{Norma Brasileira}
\sigla{ORG}{ Organização}
\sigla{PHP}{Hypertext Pre-processor}
\sigla{TCC}{Trabalho de Conclusão de Curso}
\end{ListaAbrevESiglas}

\begin{ListaDeSimbolos} % ------------------------------------------------------
\end{ListaDeSimbolos}

\begin{Sumario} % --------------------------------------------------------------
\end{Sumario}

\end{ElementosPreTextuais}
% ==================================================================================

% ============================================================================
% INTRODUÇÃO - EDITE O TEXTO ABAIXO
% ============================================================================
\begin{ElementosTextuais}
\begin{Introducao} % ---------------------------------------------------------------
\section{Introdução}
A FAI - Centro de Ensino Superior em Gestão, Tecnologia e Educação (FAI), por meio da sua biblioteca, disponibiliza este manual para normalização de trabalhos acadêmicos para os seus alunos, professores e funcionários. Este material é essencial ao uso das normas técnicas para uma apresentação correta e melhor compreensão da leitura, uma vez que um trabalho de nível superior, ou de pós-graduação, é analisado por uma banca examinadora composta por profissionais diversos, de elevado nível de conhecimentos sobre o assunto.

A normalização adotada neste manual tem como base, as normas de documentação da Associação Brasileira de Normas Técnicas (ABNT), tais como:
\begin{alphaitemize}
    \item BNT NBR 6023:2002 - Informação e documentação - Referências - Elaboração;
    \item ABNT NBR 6024:2012 - Informação e documentação - Numeração Progressiva das
seções de um documento escrito - apresentação;
    \item ABNT NBR 6027:2012 - Informação e documentação - Sumário - apresentação;
    \item ABNT NBR 6028:2003 - Informação e documentação;
    \item ABNT NBR 6034:2004 - Informação e documentação - Índice – apresentação;
    \item ABNT NBR 10520:2002 - Informação e documentação - Citação em documentos -
apresentação;
    \item ABNT NBR 12225:2004 - Informação e documentação - Lombada – apresentação;
    \item ABNT NBR 14724:2011 - Informação e documentação - Trabalhos Acadêmicos -
apresentação.
\end{alphaitemize}

Devido à atualização dos padrões nacionais e internacionais, fez-se necessária mais uma revisão no manual, que por sua vez, encontra-se em sua 8ª edição. A apresentação gráfica dos textos científicos é regulamentada pela ABNT, pois segue o padrão básico internacional, o qual organiza e permite a identificação de formas e origens de textos científicos em todo o mundo (SANTOS, 2001). É importante ressaltar que em alguns casos a ABNT apresenta em suas normas algumas regras que são opcionais, permitindo que a instituição defina seus próprios critérios. Por isso, a FAI decidiu pela utilização de alguns critérios mencionados neste manual para promover a padronização e facilitar a compreensão da comunidade acadêmica acerca da realização de seus trabalhos acadêmico/científicos.

\end{Introducao}

% ============================================================================
% DESENVOLVIMENTO- EDITE O TEXTO ABAIXO. É O CONTEÚDO PRINCIPAL DO TRABALHO
% ============================================================================
\begin{Desenvolvimento} % ------------------------------------------------------------------

\section{Trabalhos Acadêmicos}
A NBR 14724 (ABNT, 2011) especifica os princípios gerais para a elaboração de trabalhos acadêmicos (teses, dissertações, trabalhos de conclusão de curso, monografias e outros trabalhos similares), visando sua apresentação à Instituição de Ensino, a bancas e comissões examinadoras compostas por professores, especialistas, designados e/ou outros.

\subsection{Definições}
Seguem as definições da NBR 14724 (ABNT, 2011) e da NBR 6022 (ABNT, 2003), para trabalhos acadêmicos e artigos:

\begin{alphaitemize}
    \item Dissertação: documento que apresenta o resultado de um trabalho experimental ou exposição de um estudo científico retrospectivo, de tema único e bem delimitado em sua extensão, com o objetivo de reunir, analisar e interpretar informações. Deve evidenciar o conhecimento da literatura existente sobre o assunto e a capacidade de sistematização do candidato. É feito sob a coordenação de um orientador (Doutor), visando à obtenção do título de mestre.
    \item Tese: documento que representa o resultado de um trabalho experimental ou exposição de um estudo científico de tema único e bem delimitado. Deve ser elaborado com base em investigação original, constituindo-se em real contribuição para a especialidade em questão. É feito sob a coordenação de um orientador (Doutor) e visa a obtenção do título de doutor, ou similar.
    \item Trabalho de conclusão de graduação, trabalho de graduação interdisciplinar, trabalho de conclusão de curso de especialização e/ou aperfeiçoamento: documento que apresenta o resultado de estudo, devendo expressar conhecimento do assunto escolhido, que deve ser obrigatoriamente emanado da disciplina, módulo, estudo independente, curso, programa ou outros ministrados. Deve ser feito sob a coordenação de um Professor orientador.
    \item Artigo científico: parte de uma publicação com autoria declarada, que apresenta e discute ideias, métodos, técnicas, processos e resultados nas diversas áreas do conhecimento.
\end{alphaitemize}

Santos (2001, p. 37) conceitua monografia como:
\begin{citacaodireta}
Um texto de primeira mão resultante de pesquisa científica e que contém a identificação, o posicionamento, o tratamento, e o fechamento competente de um tema/problema que essencialmente analisa, em que o objeto (o tema, o problema é geralmente bem delimitado em extensão, a monografia permite o aprofundamento do estudo). Fundamenta-se na organização e na interpretação analítica e avaliativa de dados, conforme objetivos (linhas de raciocínio) pré-estabelecidos. A matéria prima do raciocínio são os dados, que basicamente se constituem de axiomas científicos (verdadeiros aceitos por diversas ciências), da autoridade de autores consagrados, com suas ilustrações, testemunhos e, até mesmo, da sua experiência pessoal coerente de pesquisador. O raciocínio é desenvolvido de forma indutiva (parte-se de experiências e observações particulares para se chegar a um princípio geral), ou de forma dedutiva (parte-se de um principio geral para verificá-la em casos particulares).
\end{citacaodireta}

Segundo França e Vasconcelos (2008) a monografia, por ser uma primeira experiência de relato científico, constitui-se numa preparação metodologia para futuras investigações. Conforme Salomon (1979, p. 36),

\begin{quotation}
O trabalho científico é identificado, frequentemente, com a investigação científica ou com o seu resultado, quando este é comunicado. Perfeitamente válida a identificação, uma vez que dá à investigação o seu devido lugar e, ao mesmo tempo, mostra a importância da comunicação no processo de elaboração dos trabalhos científicos.
\end{quotation}

Definidos, de forma bastante resumida, os vários tipos de trabalhos científicos existentes, passa- se às informações sobre a estrutura de composição desses documentos.

\subsection{Estrutura}
A estrutura de trabalhos acadêmicos (ou científicos) tais como relatórios de estágio, artigos e monografias, compreende os seguintes elementos: parte externa, pré-textuais, textuais e pós- textuais, conforme mostra a Figura \ref{fig:EstruturaTrabalhoAcademico}.

\begin{figure}[h!]
  \centering
  \includegraphics[width=1\textwidth]{ilustracoes/figuras/Estrutura de trabalho acadêmico.png}
  \caption{Estrutura de trabalhos acadêmicos, monografias e relatórios de estágio}
  \vspace{0em}
  \begin{minipage}{\textwidth}
    FONTE: O autor
  \end{minipage}
  \label{fig:EstruturaTrabalhoAcademico}
\end{figure}


O Quadro 1 especifica os elementos em função do tipo de trabalho. Note-se que nas partes pré- textuais e pós-textuais há elementos de inclusão obrigatória, opcional e condicional (opcional, mas com indicação de inclusão em determinados casos). Já as partes textuais são obrigatórias para todos os tipos de documentos acadêmico-científicos.

\section{Regras gerais de apresentação}
Os trabalhos acadêmicos devem ser apresentados formalmente conforme as especificações descritas nas seções a seguir. Ver síntese no Apêndice A.

\subsection{Formato}
Os textos devem ser apresentados em papel branco, formato A4 (21cm x 29,7cm) digitados na cor preta, com exceção das ilustrações, que podem ser coloridas. O documento deve ser produzido usando-se apenas um lado do papel. Livros e periódicos cujos formatos são definidos pela editora constituem exceções a essa regra, conforme NBR 14724 (ABNT, 2011).

As folhas devem apresentar margens superior e esquerda de 3 cm e inferior e direita de 2 cm. Recomenda-se, para digitação, utilizar a fonte Times New Roman, tamanho 12 para o texto e tamanho menor (opta-se por tamanho 10) para citações de mais de três linhas, notas de rodapé, paginação e legendas das ilustrações e tabelas. Utiliza-se o estilo itálico para nomes científicos e expressões estrangeiras, caso ocorram no texto.

Todas as folhas, incluindo-se a capa, folha de rosto, folha de aprovação, dedicatória, agradecimentos, epígrafe, resumo, abstract, listas, sumário, são contadas e numeradas com romanos minúsculos, ou não numeradas. Os números de páginas não são visualizados na capa, nem na primeira página de cada capítulo, conforme NBR 14724 (ABNT, 2011).

Existe um modelo-padrão segundo as NBR 6024 (ABNT, 2012) e NBR 14724 (ABNT, 2011) para se colocar a numeração das páginas: a visualização do número da página, em algarismos arábicos, começa a partir da segunda folha da parte textual (Introdução) no canto superior direito da folha, usando fonte Times New Roman número 10.

Os elementos textuais (Introdução, Desenvolvimento e Conclusão) e pós-textuais (referências, obras consultadas, glossário, apêndice e anexo) são contados e numerados (isto é, a numeração da página é visualizada). Quando o trabalho for composto por 2 (dois) ou mais volumes, mantém-se uma única sequência de numeração das folhas, do primeiro ao último volume. As folhas dos apêndices e anexos devem ser numeradas de maneira contínua e sua paginação dá seguimento à do texto principal.

Obs.: Trabalhos de monografias devem ter um mínimo de trinta (30) e um máximo de sessenta (60) páginas, incluindo anexos; e os artigos um máximo de 20 páginas.

Todo o texto é digitado com espaçamento 1,5 entre linhas, com 12 pontos antes e 12 pontos depois de cada parágrafo, excetuando-se as citações de mais de três linhas, notas de rodapé, legendas das ilustrações, tabelas e referências, a natureza do trabalho, o objetivo, o nome da Instituição a que é submetida e a área de concentração (que se apresentam na folha de rosto), que devem ser digitados em espaço simples. As referências, ao final do trabalho, devem utilizar entrelinhas simples separadas entre si por uma linha em branco.

Quanto à separação entre títulos, subtítulos e texto, observa-se o seguinte:

\begin{alphaitemize}
  \item os títulos de capítulos estão sempre no início de uma nova página; não há texto antes deles, portanto;
  \item separação entre títulos e subtítulos de capítulos e texto: entre linhas de 1,5 com espaçamento de 12 pontos antes e depois, exatamente como a separação de parágrafos;
  \item formatação do texto, incluindo-se títulos e subtítulos: entre linhas de 1,5 com 12 pontos antes e 12 pontos depois;
  \item separação entre o texto e os subtítulos: pula-se uma linha em branco.
\end{alphaitemize}

Conforme NBR 10520 (ABNT, 2002), para a paragrafação de um trabalho científico podem ser utilizados dois estilos: o parágrafo sem recuo ou com recuo da margem. No caso de se optar pelo parágrafo com recuo, a primeira linha do parágrafo terá recuo de 1,25 cm da margem esquerda. No caso de se optar pelo parágrafo sem recuo, o modelo adotado pela FAI, ele deve ser justificado, separando-se os parágrafos com 12 pontos antes e 12 pontos depois, exatamente como o texto já se encontra formatado. Qualquer que seja a modalidade escolhida, deve-se utilizar o mesmo estilo de paragrafação no documento inteiro.

As formas abreviadas de nomes abreviaturas e siglas são usadas para evitar a repetição de palavras e expressões frequentemente utilizadas no texto (FRANÇA; VASCONCELOS, 2008). Quando for usar sigla no texto do trabalho, deve-se observar o seguinte: a primeira vez que a sigla aparecer no texto, ela deve vir precedida pelo nome completo por extenso e, em seguida, deve-se colocá-la entre parênteses;

Exemplos: Associação Brasileira de Normas Técnicas (ABNT)
Instituto Brasileiro de Geografia e Estatística (IBGE)
Fundação Vale do Taquari de Educação e Desenvolvimento Social (FUVATES)

Quando a obra tiver mais de um volume, todos os volumes deverão apresentar folha de rosto, destacando-se a indicação “Volume I” e “Volume II” logo abaixo do título ou subtítulo, se houver. A numeração das folhas dos outros volumes deve ser uma sequência natural do primeiro volume.

\subsection{Equações e Fórmulas}
Quando houver equações e fórmulas no texto, elas devem aparecer destacadas, de modo a facilitar a sua leitura. Na sequência normal do texto é permitido o uso de uma entrelinha maior que comporte seus elementos (expoentes, índice e outros).

Quando as equações e fórmulas vierem destacadas do parágrafo são centralizadas e, se necessário, deve-se numerá-las. Caso venham fragmentadas em mais de uma linha, por falta de espaço, devem ser interrompidas antes do sinal de igualdade ou depois dos sinais de adição, subtração, multiplicação, ou divisão.

Exemplo: As chamadas equações e fórmulas no texto devem ser feitas da seguinte forma:
\begin{equation}
x^2 + y^2 = z^2 \label{eq:pythagoras}
\end{equation}

\begin{equation}
(x^2 + y^2) = n \label{eq:pythagoras2}
\end{equation}

\begin{equation}
x = \frac{-b \pm \sqrt{b^2-4ac}}{2a}
\end{equation}

\begin{equation}
(1 + x)^n = 1 + \frac{nx}{1!} + \frac{n(n-1)x^2}{2!} + \cdots
\end{equation}

\end{Desenvolvimento}

\begin{Conclusao} % ------------------------------------------------------------------

\section{Conclusão}
...

\end{Conclusao}

% ===================================================================================
\end{ElementosTextuais}

% ====== ELEMENTOS PÓS-TEXTUAIS ===================================================
\begin{ElementosPosTextuais}
\begin{Referencias} % ----------------------------------------------------------------
\end{Referencias}

\begin{Glossario} % ----------------------------------------------------------------
\end{Glossario}

\begin{Apendice} % -----------------------------------------------------------------
\end{Apendice}

\begin{Anexo} % --------------------------------------------------------------------
\end{Anexo}

\begin{Indice} % --------------------------------------------------------------------
\end{Indice}

% ===================================================================================
\end{ElementosPosTextuais}

\end{document}