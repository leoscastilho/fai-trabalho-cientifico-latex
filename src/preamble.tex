%Referência de Modelo para a FAI: https://www.fai-mg.br/biblio/images/publicacoes/modelo/ModeloArtigoCientifico.pdf %

\usepackage[brazil]{babel}  % Usa o idioma português (Brasil) para definir as convenções de idioma, como a separação de sílabas e tradução de termos.
\usepackage[utf8]{inputenc}  % Define a codificação de entrada como UTF-8 para permitir caracteres acentuados corretamente.
\usepackage[T1]{fontenc}  % Define a codificação de saída de fontes como T1, permitindo o uso correto de caracteres especiais.
\usepackage{times}     % Fonte Times New Roman
\usepackage{parskip}   % Remove a identação dos parágrafos
\usepackage{geometry}  % Permite personalizar as margens e o layout da página.
\usepackage{graphicx}  % Permite inserir gráficos, imagens e ajustar o tamanho deles.
\usepackage{titlesec}  % Fornece comandos para personalizar o formato dos títulos das seções.
\usepackage{fancyhdr}  % Permite personalizar os cabeçalhos e rodapés do documento.
\usepackage{setspace}  % Controla o espaçamento entre linhas no documento (aqui é usado para espaçamento de 1,5 linha).
\usepackage{csquotes}  % Fornece suporte para citações, garantindo que o estilo de aspas seja correto em diferentes idiomas.
\usepackage{ragged2e}  % Fornece comandos para justificar ou ajustar o alinhamento de texto.
\usepackage{lipsum}    % Para gerar texto Lorem Ipsum
\usepackage{enumitem} % Customizar alíneas
\usepackage[toc,page]{appendix} % Para gerenciar apêndices
\usepackage{pgfmath} % Para cálculos matemáticos nos apêndices

\setlength{\parindent}{0pt}  % Paragráfos sem identação, conforme modelo da FAI
\setstretch{1.5}  % This also sets 1.5 line spacing
\setlength{\parskip}{12pt}  % Espaçament ode 12pt entre paragrafos

% Define a numeração da página
\fancypagestyle{numbered}{
    \fancyhf{}
    \fancyhead[R]{{\fontsize{10pt}{12pt}\selectfont\thepage}}
    \renewcommand{\headrulewidth}{0pt}
}

% Quanto à separação entre títulos, subtítulos e texto, observa-se o seguinte:
\let\oldsection\section
\renewcommand{\section}[1]{%
    \clearpage % os títulos de capítulos estão sempre no início de uma nova página;
    \thispagestyle{empty} % No page number on this page
    \oldsection{#1}
}
% seções primárias: os títulos dessas seções devem ser em caixa alta, fonte 12 e em negrito;
\titleformat{\section}{\normalfont\fontsize{12}{12}\selectfont\bfseries}{\thesection}{0.3em}{\MakeUppercase}

\let\oldsubsection\subsection
\renewcommand{\subsection}[1]{%
    \par\vspace{2\parskip}
    \oldsubsection{#1}
    \par\vspace{0.5\parskip} % separação entre o texto e os subtítulos: pula-se uma linha em branco.
}
% seções secundárias: os títulos dessas seções devem ser iguais aos das primárias, porém sem negrito;
\titleformat{\subsection}{\normalfont\fontsize{12}{12}\selectfont}{\thesubsection}{0.3em}{\MakeUppercase}

% Define spacing for subsubsection
\let\oldsubsubsection\subsubsection
\renewcommand{\subsubsection}[1]{%
    \par\vspace{2\parskip}
    \oldsubsubsection{#1}
    \par\vspace{0.5\parskip}% Add paragraph break after
}
% seções terciárias: os títulos dessas seções devem ser em negrito, somente com a primeira letra maiúscula;
\titleformat{\subsubsection}{\normalfont\fontsize{12}{14.4}\bfseries}{\thesubsubsection}{0.3em}{}

\geometry{
    left=3cm,
    right=2cm,
    top=3cm,
    bottom=2cm,
    headsep=35pt,
    footskip=0pt
    % , showframe
} % Define as margens da página (esquerda = 3cm, direita = 2cm, topo = 3cm, fundo = 2cm)

% Configuração das alíneas
\newlist{alinea}{enumerate}{1}
\setlist[alinea,1]{
    label=\alph*), % c) as alíneas devem ser indicadas alfabeticamente, em letra minúscula, seguida de parêntese fechado. Utilizam-se letras dobradas, quando esgotadas as letras do alfabeto;
    leftmargin=1cm,        % Total indentation from current margin
    itemindent=0pt,        % No additional item indentation
    labelsep=0pt,          % No space between label and text
    labelwidth=13pt,       % d) as letras indicativas das alíneas devem apresentar recuo de 0,5 cm em relação à margem esquerda;
    itemsep=0pt,
    parsep=0pt,
    topsep=0pt,
    partopsep=0pt,
    align=left,
    before={\setlength{\parskip}{0pt}},  % h) Entre as alíneas, inclusive a primeira delas, os parágrafos são formatados com zero ponto ante e zero depois
    after={\setlength{\parskip}{12pt}}   % i) O próximo parágrafo, após a última alínea, volta a ter formatação normal com 12 pontos antes e 12 depois.
}

\usepackage[sorting=nyt]{biblatex}  % Carrega o pacote BibLaTeX para bibliografia e define o estilo de ordenação como "nyt" (nome, ano, título).
\renewcommand*{\bibfont}{\raggedright}  % Alinha a bibliografia à esquerda (de forma justificada).

% Redefine o título da seção de referências para maiúsculas e centralizado
\defbibheading{bibliography}{%
  \begin{center}
    \textbf{\MakeUppercase{Referências}}
  \end{center}
}

% Customização do Sumário ----------------------------------------------------------
\usepackage{tocloft}
\usepackage{xstring}
\usepackage{textcase}

\renewcommand{\contentsname}{SUMÁRIO}
\renewcommand{\cftsecfont}{\bfseries}
\renewcommand{\cftsecpagefont}{\bfseries}
\renewcommand{\cftsecleader}{\cftdotfill{\cftdotsep}}
\setlength{\cftbeforetoctitleskip}{0pt}
\setlength{\cftaftertoctitleskip}{0.4cm}
\renewcommand{\cfttoctitlefont}{\hfil\bfseries}


\renewcommand{\cfttoctitlefont}{\hfil\rmfamily\bfseries\MakeUppercase} % Título maiúsculo
\renewcommand{\cftdotsep}{0.1} % Adiciona mais pontos. Menor numero = Mais pontos
\setlength{\cftbeforesecskip}{0em} % Espaço entre as linhas de seção
\setlength{\cftbeforesubsecskip}{0em} % Espaço entre as linhas de subseção
\setlength{\cftbeforesubsubsecskip}{0em} % Espaço entre as linhas de subsubseção
\setlength{\cftsecindent}{0em} % Remove identação das seções
\setlength{\cftsubsecindent}{0em}   % Remove identação das subseções
\setlength{\cftsubsubsecindent}{0em} % Remove identação das subsubseções

\setlength{\cftsecnumwidth}{1.2em} % Espaço entre número da seção e texto
\setlength{\cftsubsecnumwidth}{1.5em} % Espaço entre número da subseção e texto
\setlength{\cftsubsubsecnumwidth}{2.3em} % Espaço entre número da subsubseção e texto

% Seção, subseção e subsubseção em maiúsculo no Sumário
\makeatletter
\let\oldcontentsline\contentsline
\renewcommand{\contentsline}[4]{%
  \expandafter\ifx\csname l@#1\endcsname\l@section
    \oldcontentsline{#1}{\MakeUppercase{#2}}{#3}{#4}%
  \else
    \expandafter\ifx\csname l@#1\endcsname\l@subsection
      \oldcontentsline{#1}{\MakeUppercase{#2}}{#3}{#4}%
    \else
      \expandafter\ifx\csname l@#1\endcsname\l@subsubsection
        \oldcontentsline{#1}{\MakeUppercase{#2}}{#3}{#4}%
      \else
        \oldcontentsline{#1}{#2}{#3}{#4}%
      \fi
    \fi
  \fi
}
\makeatother
% ------------------------------------------------------------------------------------

% Funções customizadas para criar e imprimir autores ---------------------------------
\usepackage{pgffor}

\newcounter{authorcount} % Counter para autores

% Comando para adicionar novos autores
\newcommand{\AddAuthor}[2]{% #1 = nome, #2 = email
    \stepcounter{authorcount}
    \expandafter\def\csname author\theauthorcount\endcsname{#1}
    \expandafter\def\csname email\theauthorcount\endcsname{#2}
}

% Comando para imprimir os autores
\newcommand{\PrintAuthorsLombada}{%
    \foreach \i in {1,...,\theauthorcount} {%
        \textbf{\fontsize{12}{14}\selectfont\MakeUppercase{\csname author\i\endcsname}} \\
    }%
}
% ------------------------------------------------------------------------------------

% Funções para Citação Direta ---------------------------------
\newenvironment{citacaodireta}
{\par\addvspace{\baselineskip}% Add exactly one line space before
 \begingroup
 \list{}{\leftmargin=4cm
         \rightmargin=0pt
         \parsep=0pt
         \itemsep=0pt
         \topsep=0pt}
 \item\relax
 \singlespacing
 \fontsize{10}{12}\selectfont
 \justifying}
{\endlist
 \endgroup
 \par\addvspace{\baselineskip}}% Add exactly one line space after

% Usar:
% \begin{citacaodireta}
% ...
% \end{citacaodireta}
% ------------------------------------------------------------------------------------

% Customização das Listas de Ilustrações ---------------------------------
\usepackage[labelfont=bf]{caption}
\usepackage{newfloat} % Para criar novos tipos de float

% Definir novos tipos de float
\DeclareFloatingEnvironment[
    fileext=lof,
    listname={Lista de Figuras},
    name=Figura,
    placement=htbp,
]{figura}

\DeclareFloatingEnvironment[
    fileext=lop,
    listname={Lista de Fotografias},
    name=Fotografia,
    placement=htbp,
]{fotografia}

\DeclareFloatingEnvironment[
    fileext=grf,
    listname={Lista de Gráficos},
    name=Gráfico,
    placement=htbp,
]{grafico}

\DeclareFloatingEnvironment[
    fileext=loq,
    listname={Lista de Quadros},
    name=Quadro,
    placement=htbp,
]{quadro}

% Contadores para cada tipo
\newcounter{totalfiguras}
\newcounter{totalfotografias}
\newcounter{totalgraficos}
\newcounter{totalquadros}

% Comandos para incrementar contadores
\newcommand{\addfigura}{\stepcounter{totalfiguras}}
\newcommand{\addfotografia}{\stepcounter{totalfotografias}}
\newcommand{\addgrafico}{\stepcounter{totalgraficos}}
\newcommand{\addquadro}{\stepcounter{totalquadros}}

% Sistema de duas passadas para listas condicionais
\makeatletter
% Escrever contadores no arquivo auxiliar ao final do documento
\AtEndDocument{%
    \immediate\write\@auxout{\string\gdef\string\@totalfiguras{\arabic{totalfiguras}}}%
    \immediate\write\@auxout{\string\gdef\string\@totalfotografias{\arabic{totalfotografias}}}%
    \immediate\write\@auxout{\string\gdef\string\@totalgraficos{\arabic{totalgraficos}}}%
    \immediate\write\@auxout{\string\gdef\string\@totalquadros{\arabic{totalquadros}}}%
}

% Definir valores padrão se não existirem (primeira compilação)
\@ifundefined{@totalfiguras}{\gdef\@totalfiguras{0}}{}
\@ifundefined{@totalfotografias}{\gdef\@totalfotografias{0}}{}
\@ifundefined{@totalgraficos}{\gdef\@totalgraficos{0}}{}
\@ifundefined{@totalquadros}{\gdef\@totalquadros{0}}{}

% Comandos condicionais usando os valores da compilação anterior
\newcommand{\ifhasfiguras}[2]{\ifnum\@totalfiguras>0 #1\else #2\fi}
\newcommand{\ifhasfotografias}[2]{\ifnum\@totalfotografias>0 #1\else #2\fi}
\newcommand{\ifhasgraficos}[2]{\ifnum\@totalgraficos>0 #1\else #2\fi}
\newcommand{\ifhasquadros}[2]{\ifnum\@totalquadros>0 #1\else #2\fi}
\makeatother
% ------------------------------------------------------------------------------------

% Customização da Lista de Figuras ---------------------------------
\usepackage[labelfont=bf]{caption}
\usepackage[figure]{totalcount} % Add this package
\makeatletter
\renewcommand{\@cftmakeloftitle}{} % Completely remove title generation
\makeatother
% Configure the figure entries with uppercase and dash
\renewcommand{\cftfigpresnum}{FIGURA }
\renewcommand{\cftfigaftersnum}{ -- }
\setlength{\cftfignumwidth}{2.5cm} % Adjust width for new format
\setlength{\cftfigindent}{0pt} % No indentation

% Note: List formatting for other illustration types (fotografia, grafico, quadro)
% is handled by the newfloat package automatically. The caption formatting
% below will ensure consistent appearance in both captions and lists.
% ------------------------------------------------------------------------------------

\usepackage{caption}

\newcommand{\fonte}[1]{%
  \vspace{0em}%
  \begin{minipage}{\textwidth}%
    \raggedright%
    \footnotesize%
    FONTE: #1%
  \end{minipage}%
}

% Create \legenda as alias for \caption with full functionality
\NewCommandCopy{\legenda}{\caption}

% Create \equacao as alias for \equacao with full functionality
\newenvironment{equacao}{\begin{equation}}{\end{equation}}

% Custom caption formatting for all illustration types
% Using footnote size, capital letters, and dash separator
\captionsetup[quadro]{
    justification=raggedright,
    singlelinecheck=false,
    font={footnotesize},
    labelfont={footnotesize},
    labelformat=simple,
    labelsep=none,
    format=plain
}

\captionsetup[figura]{
    justification=raggedright,
    singlelinecheck=false,
    font={footnotesize},
    labelfont={footnotesize},
    labelformat=simple,
    labelsep=none,
    format=plain
}

\captionsetup[grafico]{
    justification=raggedright,
    singlelinecheck=false,
    font={footnotesize},
    labelfont={footnotesize},
    labelformat=simple,
    labelsep=none,
    format=plain
}

\captionsetup[fotografia]{
    justification=raggedright,
    singlelinecheck=false,
    font={footnotesize},
    labelfont={footnotesize},
    labelformat=simple,
    labelsep=none,
    format=plain
}

% Custom label formatting to use uppercase names with dashes
\DeclareCaptionLabelFormat{customdash}{\MakeUppercase{#1} #2 - }

% Apply custom label format to all illustration types
\captionsetup[quadro]{labelformat=customdash}
\captionsetup[figura]{labelformat=customdash}
\captionsetup[grafico]{labelformat=customdash}
\captionsetup[fotografia]{labelformat=customdash}


\addbibresource{bib/references.bib}  % Adiciona o arquivo de bibliografia (bib/references.bib) como fonte para as referências bibliográficas.
